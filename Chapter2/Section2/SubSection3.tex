\subsection{入力速度倍率の推定}
提案手法における条件\eqrefc{eq:approximation:condition1:result} - \eqrefc{eq:approximation:condition3:result}では, 入力速度倍率$a$を推定する必要がある.
ここでは, 本手法における入力速度倍率の推定方法について述べる.

まず, 入力速度倍率$a$は, 時刻$t$における入力スパイク$o(t)$が速度変化によって時刻$at$に移動することに対応する (\figref{fig:input:speed:change}).
次に, 入力スパイク密度$fr$を定義する.
入力スパイク密度$fr$とは, 単位時間当たりのスパイクの数を表す (\eqrefc{eq:input:spike:density}).
\begin{align}
    fr = \frac{1}{T} \sum_{t=1}^{T} o(t) \label{eq:input:spike:density}
\end{align}
ここで, $T$は時系列データの長さである.
また, 入力速度変化によってその時系列データの情報量 (=スパイクの総数) は変化しないと仮定する.
このことから, 入力速度倍率$a$は, 学習時の基準入力スパイク密度$fr_{base}$と推論時の入力スパイク密度$fr_{in}$の比として推定することができる.
\begin{align}
    a = \frac{fr_{base}}{fr_{in}}
\end{align}

\begin{figure}[htbp]
    \centering
    \includesvg[width=1.0\textwidth, inkscapelatex=false]{dummy/dummy_img}
    \caption{入力速度変化による入力スパイクの移動}
    \label{fig:input:speed:change}
\end{figure}


