\section{考察}

節\ref{sec:result3}では, 提案手法を用いたSNNのロボットマニピュレータ軌道予測問題におけるモデル精度評価を行った.
結果として, 提案手法を用いたSNNは未学習の速度に対して, 通常のSNNよりも目標軌道に近い軌道を予測することが可能であった.
この結果は, 提案手法によって入力の速度変化に応じてSNNの内部状態をタイムスケーリングに近似することで, 予測軌道も単純なタイムスケーリングとなったからだと考えられる.
そこで, 目標軌道速度倍率$a$を変化させたときの, SNNの内部状態の変化と提案手法の有無の関係を調べる.
% \figref{fig:discussion3:snn:a3} - \figref{fig:discussion3:dyna:a05}はそれぞれ, 学習時の速度と比較して, 入力速度が低速倍率$a=3.0$と高速倍率$a=0.5$の場合のSNNの内部状態変化を示す.
% 横軸は時間 (s) を表す.
% 最上段のグラフはマニピュレータの各関節角度$\bm{q}$を表す.
% 50ステップ (3.5 s) 過去までの関節角度軌道$\bm{q}$がThreshold Encodingを通して, SNN特徴抽出器$f^{SNN}_\theta$に入力される.
% 2段目のグラフは目標軌道のタイムスケール$a$を表す.
% タイムスケールは$a=1.0$を初期値として, 1秒間かけて$a=3.0$と$a=0.5$に線形補間によって変化させた.
% 3段目のグラフはSNNの最終層の内部状態を表す.
% 各グラフはSNN特徴抽出器$f^{SNN}_\theta$の12次元の出力に対応する.
% ここで, 提案手法においては, $a$でスケーリングした$\bm{{v'}^{t,L}}$である.
% この内部状態が予測器$f^{NN}_\theta$に入力され, 目標手先位置変化量が予測される.
% 最終段のグラフは, マニピュレータの手先位置$x,y$の軌道を表す.

目標軌道のタイムスケールが変化する前の$a=1.0$の区間では, 通常のSNN, 提案手法のSNNともに, マニピュレータの手先軌道が周期的に変化していることがわかる.
さらに, SNNの内部状態も同様に周期的に変化していることがわかる.
このことから, 学習データの8の字軌道の周期に合わせて, 内部状態も周期的に変化するようにSNNが学習されていると考えられる.
一方で, 目標軌道のタイムスケールが$a=1.0$から変化すると, 通常のSNNでは手先軌道が安定せず, $a=1.0$のときと比べて, 大きくはずれた軌道をとることがわかる.
低速倍率 (\figref{fig:discussion3:snn:a3}) の場合, 時間経過に従って, SNN内部状態の時間変化が小さくなることがわかる.
また, 高速倍率 (\figref{fig:discussion3:snn:a05}) の場合, タイムスケールが変化したのにも関わらず, 内部状態の周期は$a=1.0$のときからの変化が小さい.
このように, 通常のSNNでは, タイムスケール変化による入力速度変化に応じて, SNNの内部状態の周期が変化していない.
結果として, 通常のSNNは, 未学習の速度の手先軌道予測が困難であったと考えられる.
しかしながら, 提案手法のSNNを用いた手先軌道予測では, タイムスケール変化によって, その周期のみが安定的に変化していることがわかる.
また, 提案手法のSNNの内部状態も, その周期のみが安定的に変化していることがわかる.
目標速度が$a=3.0$の低速 (\figref{fig:discussion3:dyna:a3}) のとき, 提案手法のSNNにおける内部状態の周期は, $a=1.0$から3.0倍程度に変化している.
また, 目標速度が$a=0.5$の高速 (\figref{fig:discussion3:dyna:a05}) のとき, 提案手法のSNNにおける内部状態の周期は, $a=1.0$から0.5倍程度に変化している.
これは, 時定数を動的に変化させることで, 内部状態の時間変化量をタイムスケールに応じて変化させることができたからだと考えられる.
結果として, 特徴量である内部状態を入力とした$f^{NN}_\theta$による予測も, その振幅は変えず, 速度のみをスケールした軌道を予測することが可能であったと考えられる.

\begin{figure}[htbp] %画像形式はこのチャプターのreadmeを参照
    \centering
    \includesvg[width=1.0\textwidth, inkscapelatex=false]{Static/disc3_snn3}
    \caption{
        軌道予測における通常のSNN最終層の内部状態変化 (低速倍率変化)
        }
    \label{fig:discussion3:snn:a3}
\end{figure}

\begin{figure}[htbp] %画像形式はこのチャプターのreadmeを参照
    \centering
    \includesvg[width=1.0\textwidth, inkscapelatex=false]{Static/disc3_dyna3}
    \caption{
        軌道予測における提案手法のSNN最終層の内部状態変化 (低速倍率変化)
        }
    \label{fig:discussion3:dyna:a3}
\end{figure}


\begin{figure}[htbp] %画像形式はこのチャプターのreadmeを参照
    \centering
    \includesvg[width=1.0\textwidth, inkscapelatex=false]{Static/disc3_snn0.5}
    \caption{
        軌道予測における通常のSNN最終層の内部状態変化 (高速倍率変化)
        }
    \label{fig:discussion3:snn:a05}
\end{figure}

\begin{figure}[htbp] %画像形式はこのチャプターのreadmeを参照
    \centering
    \includesvg[width=1.0\textwidth, inkscapelatex=false]{Static/disc3_dyna0.5}
    \caption{
        軌道予測における提案手法のSNN最終層の内部状態変化 (高速倍率変化)
        }
    \label{fig:discussion3:dyna:a05}
\end{figure}
