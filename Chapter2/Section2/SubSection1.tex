\subsection{提案手法 概要}
提案手法の概要図を\ref{fig:proposed_method}に示す.
本手法では, 入力スパイク列$\bm{o}(t)$の未学習の速度に対して, その推論精度が低下しにくいSNNを実現することを目指す.
一般的に, ニューラルネットワークに対する入力の加減速に対して, その内部状態は単純な加減速とはならない.
そのため, ニューラルネットワークの最終的な出力は, 入力の速度変化に依存して変化する.
そこで本手法では, 入力の速度変化に応じて, SNNの内部状態の変化量を動的に調整する機構を導入する.
この機構によって, 入力速度変化による内部状態の変化を抑制し, 頑健な推論を行うSNNを構築する.

\begin{figure}[htbp]
    \centering
    \includesvg[width=1.0\textwidth, inkscapelatex=false]{dummy/dummy_img}
    \caption{提案手法の概要図}
    \label{fig:proposed_method}
\end{figure}

提案手法の導出は以下の3ステップで行う.
\begin{enumerate}
    \item 入力スパイク列の速度変化に対する理想的なSNNの内部状態の定式化
    \item 入力スパイク列の速度変化に対する実際のSNNの内部状態の定式化
    \item 実際の内部状態を理想的な内部状態に近似するための条件の導出
\end{enumerate}
ここで, 理想的なSNNの内部状態とは, 入力スパイク速度が$a$倍になった場合に, 内部状態の変化速度も$a$倍になることを意味する.

