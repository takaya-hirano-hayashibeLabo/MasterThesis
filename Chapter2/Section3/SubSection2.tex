\subsection{実験2 ジェスチャー分類問題における入力速度変化に対するモデル精度評価}
提案手法をジェスチャー動画分類問題に適用し, 未学習の入力速度に対するモデル精度評価を行う.

\subsubsection{データセット}
ジェスチャー動画のデータセットとして, SNNの評価で広く用いられる\cite{massa2020efficient}DVSGesture\cite{dvsgesture}を使用した.
DVSGestureはイベントベースビジョンセンサ (DVS128 : \figref{fig:dvs128}) で記録されている.
このセンサでは各ピクセルの輝度変化を非同期に捉え, その変化をイベント$\epsilon$として記録する (\eqrefc{eq:dvs:event}).
\begin{equation}
    \epsilon = (x, y, t, p) \label{eq:dvs:event}
\end{equation}
ここで, $(x, y)$はピクセルの座標, $t$はイベントが発生した時刻である.
また, $p$はイベント強度を表し, ピクセルの輝度変化が正であれば1, 負であれば-1の値を持つ.

イベントベースビジョンセンサによるジェスチャー記録の様子を\figref{fig:dvs:recordview}に示す.
上段が通常のフレームベースカメラで記録したもの, 下段がイベントベースビジョンセンサで記録したものである.
また, 黒色は値が0であることを表し, 青色, 桃色はそれぞれ$p=1, p=-1$のイベントを表す.
イベントは時間的なピクセルの輝度変化を検出するため, ジェスチャーでは動きの多い腕周りのピクセル情報が多く記録される.
\begin{figure}[htbp]
    \centering

    \begin{minipage}{0.3625\textwidth}
        \centering
        \includesvg[width=1.0\textwidth, inkscapelatex=false]{Static/chap2_sec3_dvs128}
        \caption[DVS128の外観]{DVS128\cite{dvs128fig}}
        \label{fig:dvs128}
    \end{minipage}
    \hspace{0.02\textwidth}
    \begin{minipage}{0.53746\textwidth}
        \centering
        \includesvg[width=1.0\textwidth, inkscapelatex=false]{Static/chap2_sec3_recordview}
        \caption[DVSGesture記録の様子]{DVSGesture記録の様子\cite{dvsgesture}}
        \label{fig:dvs:recordview}
    \end{minipage}
\end{figure}

DVSGestureは11種類のクラスのジェスチャーを記録している.
それぞれのジェスチャーのスナップショットを\figref{fig:dvs:gesture}に示す.
%Gestureのスナップショット
\begin{figure}[htbp]
    \centering

    \begin{minipage}{0.45\textwidth}
        \centering
        \includesvg[width=1.0\textwidth, inkscapelatex=false]{Static/gesture_snapshot/gesture_0/label0}
        \subcaption*{1. Hand clapping}
    \end{minipage}
    \hspace{0.01\textwidth}
    \begin{minipage}{0.45\textwidth}
        \centering
        \includesvg[width=1.0\textwidth, inkscapelatex=false]{Static/gesture_snapshot/gesture_1/label1}
        \subcaption*{2. Right hand wave}
    \end{minipage}

    \begin{minipage}{0.45\textwidth}
        \centering
        \includesvg[width=1.0\textwidth, inkscapelatex=false]{Static/gesture_snapshot/gesture_2/label2}
        \subcaption*{3. Left hand wave}
    \end{minipage}
    \hspace{0.01\textwidth}
    \begin{minipage}{0.45\textwidth}
        \centering
        \includesvg[width=1.0\textwidth, inkscapelatex=false]{Static/gesture_snapshot/gesture_3/label3}
        \subcaption*{4. Right arm clockwise}
    \end{minipage}

    \begin{minipage}{0.45\textwidth}
        \centering
        \includesvg[width=1.0\textwidth, inkscapelatex=false]{Static/gesture_snapshot/gesture_4/label4}
        \subcaption*{5. Right arm counterclockwise }
    \end{minipage}
    \hspace{0.01\textwidth}
    \begin{minipage}{0.45\textwidth}
        \centering
        \includesvg[width=1.0\textwidth, inkscapelatex=false]{Static/gesture_snapshot/gesture_5/label5}
        \subcaption*{6. Left arm clockwise}
    \end{minipage}

    \begin{minipage}{0.45\textwidth}
        \centering
        \includesvg[width=1.0\textwidth, inkscapelatex=false]{Static/gesture_snapshot/gesture_6/label6}
        \subcaption*{7. Left arm counterclockwise}
    \end{minipage}
    \hspace{0.01\textwidth}
    \begin{minipage}{0.45\textwidth}
        \centering
        \includesvg[width=1.0\textwidth, inkscapelatex=false]{Static/gesture_snapshot/gesture_7/label7}
        \subcaption*{8. Arm roll}
    \end{minipage}

    \begin{minipage}{0.45\textwidth}
        \centering
        \includesvg[width=1.0\textwidth, inkscapelatex=false]{Static/gesture_snapshot/gesture_8/label8}
        \subcaption*{9. Air drums}
    \end{minipage}
    \hspace{0.01\textwidth}
    \begin{minipage}{0.45\textwidth}
        \centering
        \includesvg[width=1.0\textwidth, inkscapelatex=false]{Static/gesture_snapshot/gesture_9/label9}
        \subcaption*{10. Air guitar}
    \end{minipage}

    \begin{minipage}{0.45\textwidth}
        \centering
        \includesvg[width=1.0\textwidth, inkscapelatex=false]{Static/gesture_snapshot/gesture_10/label10}
        \subcaption*{11. Other gestures }
    \end{minipage}


    \caption[データセットDVSGestureにおける各ジェスチャーのスナップショット]{
        データセットDVSGestureにおける各ジェスチャーのスナップショット
    }
    \label{fig:dvs:gesture}
\end{figure}

\subsubsection{モデルの学習}
モデルの学習はイベントの時系列データを入力, ジェスチャーのクラスを出力とする分類問題として行う.
ここで, 使用するデータセットのイベントの有無をスパイクとして扱うことでSNNの入力としている.
学習させるモデルは, 通常のSNN, Parametric-SNN, 提案手法のSNNとした.
Parametric-SNNとは, ニューラルネットワークの重みとバイアスに加えて, SNNのLIFモデルにおける時定数も学習可能としたSNNである.
モデル構成と学習時のパラメータを\tabref{tab:exp2:model}, \tabref{tab:exp2:model:parameter:lif}, \tabref{tab:exp2:train:parameter}に示す.
また, 学習させたそれぞれのモデル構成は統一で\tabref{tab:exp2:model}の値を用いており, その相違点はニューラルネットワークのバイアスと時定数の学習についてのみである (\tabref{tab:exp2:parameter}).
\begin{table}[htb]
    \centering
    \caption{ジェスチャー分類モデル構成}
    \label{tab:exp2:model}
    \begin{tabular}{ccccc}
        \hline
        \textbf{Layer}& \textbf{Type}&\textbf{Input size} & \textbf{Output size} & \textbf{Residual block nums}\\
        \hline
        1   & MS-ResNet & 2x32x32 & 12x32x32 & 3\\
        2 & AveragePooling2d & 12x32x32 & 12x16x16 & - \\
        3 & MS-ResNet & 12x16x16 & 32x16x16 & 3\\
        4 & AveragePooling2d & 32x16x16 & 32x8x8 & - \\
        5 & Linear & 2048 & 512 & - \\
        6 & Linear & 512 & 11 & - \\
        \hline
    \end{tabular}
\end{table}

\begin{table}[htb]
    \centering
    \caption{LIFモデルのパラメータ}
    \label{tab:exp2:model:parameter:lif}
    \begin{tabular}{ccccc}
        \hline
        $\bm{dt}$& $\bm{v_{rest}}$ & $\bm{v_{th}}$ & $\bm{\tau}$ & $\bm{r}$\\
        \hline
        0.003   & 0.0 & 0.1 & 0.006 & 1 \\
        \hline
    \end{tabular}
\end{table}


\begin{table}[htb]
    \centering
    \caption{モデルの学習条件}
    \label{tab:exp2:train:parameter}
    \begin{tabular}{cc}
        \hline
        学習率 $lr$ & 0.0008\\
        バッチサイズ $batch\_size$ & 64\\
        エポック数 $epoches$ & 800\\
        weight decay $wd$ & 0.001\\
        勾配クリッピング $clip\_norm$ & 1.0\\
        optimizer & AdamW\\
        \hline
    \end{tabular}
\end{table}


\begin{table}[htb]
    \centering
    \caption{各モデルの相違点}
    \label{tab:exp2:parameter}
    \begin{tabular}{ccccc}
        \hline
         & \textbf{SNN} & \textbf{Parametric-SNN} & \textbf{提案手法}\\
         \textbf{NNのバイアス$b$}&あり&あり&なし($=0$)\\
         \textbf{時定数の学習$\tau$}&なし&あり&あり\\
        \hline
    \end{tabular}
\end{table}

学習時の損失関数を\eqrefc{eq:exp2:loss}に示す.
\begin{align}
    \hat{y}_i &= \frac{1}{T} \sum_{t=1}^T o_i^t \notag \\
    \mathcal{L}_{CE} &= -\sum_{i=1}^{11} y_i \log(\hat{y}_i) \label{eq:exp2:loss}
\end{align}
ここで, $\mathcal{L}_{CE}$はクロスエントロピー損失, $y_i$はone-hotベクトル化した正解ラベルである.
また, SNNの出力$o_i^t$は時間軸を持つため, 単位時間あたりのスパイク密度を$\hat{y}_i$とし, その値をモデルのクラス推論確率としている.
推論時は$\hat{y}_i$が最大のクラスを推論結果とする決定的推論を行う (\eqrefc{eq:exp2:inference}).
\begin{equation}
\hat{y} = \mathop{\arg\max}_{i}(\hat{y}_i) \label{eq:exp2:inference}
\end{equation}



\subsubsection{評価方法}
提案手法のジェスチャ分類問題における入力速度変化に対する頑健性を評価するために以下の2つの評価を行った.
\begin{itemize}
    \item 入力シーケンス全体を$a$倍速した際のモデル精度評価
    \item 入力シーケンスの前半, 後半をそれぞれ$a_1, a_2$倍速した際のモデル精度評価
\end{itemize}
まず, 入力シーケンス全体の速度倍率を変更したときのモデル精度を評価した.
速度倍率$a$は0.1, 0.2, 0.3, 0.4, 0.5, 0.6, 0.7, 0.8, 0.9, 1.0, 2.0, 3.0, 4.0, 5.0, 6.0, 7.0, 8.0, 9.0, 10.0の20種類の値を用いた.
次に, 入力シーケンスを前半と後半に分割し, 途中で速度倍率を変更した.
このデータを入力したときのモデル精度比較することで, 途中で速度が変更される場合のモデル精度を評価した.
$a_1, a_2$はそれぞれ\tabref{tab:model:parameter:speed:change}に示す値を用いた.
ここで, $a=1.0$は学習時の速度データである.

\begin{table}[htb]
    \centering
    \caption{速度変更のパラメータ $a_1, a_2$}
    \label{tab:model:parameter:speed:change}
    \begin{tabular}{ccc}
        \hline
        \textbf{Case}& $a_1$ & $a_2$\\
        \hline
        A&1.0&3.0\\
        B&1.0&0.2\\
        C&3.0&0.2\\
        D&0.2&3.0\\
        \hline
    \end{tabular}
\end{table}
