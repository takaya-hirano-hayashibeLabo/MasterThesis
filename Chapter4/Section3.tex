\section{考察}
第\ref{sec:result2}節では, 提案手法を用いてジェスチャー分類問題を扱い, 未学習速度に対する分類精度を検証した.
その結果, 提案手法を用いたSNNは, 通常のSNNやParametric-SNNと比較して, 入力速度変化に対してモデル性能が維持されることが示された.
また, 提案手法を用いることで, 入力速度変化に伴うSNNの内部状態の活性パターンを抑制できることが示された.


% 結果として, 提案手法を用いたSNNは, 未学習の速度のジェスチャーデータに対する頑健性を示した.
% この結果は, 提案手法によって入力の速度変化に応じてSNNの内部状態をタイムスケーリングすることで, 入力速度変化によらずSNNの出力パターンが一定に保たれたからだと考えられる.
% 学習時速度$a=1.0$を入力したときの内部状態図 (\figref{fig:discussion2:snn:a5} - \figref{fig:discussion2:dyna:a05}における上段の図) より, 通常のSNNも提案手法のSNNも正解ラベルのニューロンが最も活性化しており, 正しく推論ができていることがわかる.
% 一方で入力速度が変化すると, 通常のSNNはその内部状態パターンが大きく変化し, 誤った推論を行っていることがわかる.
% 低速倍率の入力 (\figref{fig:discussion2:snn:a5}) の場合, 正解であるクラス3のニューロンの活性化が小さくなり, クラス8のニューロンの活性化が大きくなっている.
% また, 高速倍率の入力 (\figref{fig:discussion2:snn:a05}) の場合, 正解であるクラス1のニューロンの活性化が小さくなり, クラス3, クラス4のニューロンの活性化が大きくなっている.
% このような内部状態の挙動がみられる原因としては, 入力の速度変化によって1ステップあたりの情報量が変化したのにも関わらず, 通常のSNNの内部状態は学習時の速度に依存した変化をしていることが考えられる.
% 結果として, 内部状態の活性化パターンが入力速度変化に依存し, 誤った推論を行ったと考えられる.
% しかしながら, 提案手法のSNNは, 入力速度変化による内部状態パターンの変化が小さく, 正しく推論を行っていることがわかる.
% これは, 提案手法を用いることで, 入力速度変化に従って内部状態の変化量を動的に変化させているからだと考えられる.
% 特に, この役割は提案手法の条件\eqrefc{eq:approximation:condition1:result}が担うと考えられる.
% 入力が低速倍された場合 (\figref{fig:discussion2:dyna:a5}), SNNの時定数は提案手法の条件\eqrefc{eq:approximation:condition1:result}によって, 学習時よりも大きな値となる.
% これによって, 1ステップあたりの内部状態の忘却量は小さくなり, より長い時間をかけてで内部状態が変化する.
% また, 入力が高速倍された場合 (\figref{fig:discussion2:dyna:a05}) は, 反対にSNNの時定数は学習時よりも小さな値となる.
% これによって, 1ステップあたりの内部状態の忘却量は大きくなり, より短い時間の中で内部状態が変化する.
% このように, 提案手法を用いることで, 入力速度変化に伴ってSNNの内部状態の変化量を動的に変化させることができる.
% さらに, 時定数の変化は内部状態が単純なタイムスケーリングに近似する条件となっている.
% 結果として, 未学習の入力速度に対しても, SNNの内部状態が学習時と同様のパターンで振る舞い, 正しい推論が可能になったと考えられる.


\clearpage