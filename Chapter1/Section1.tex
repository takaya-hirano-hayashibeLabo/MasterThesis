\section{研究背景}
Spiking Neural Network (SNN)は, 通常のArtificial Neural Network (ANN)と比較して, より脳神経回路を模した数理モデルである.
脳における電気パルス信号を介した神経ダイナミクスを数理モデルとして持つため, SNNはその生物的妥当性が高い.
また, SNNをニューロモーフィックデバイスを用いて実装することでその消費電力を削減できることも知られており, ロボットの行動計画や物体認識, 言語モデルなどの幅広いタスクにおいてその有効性が示されている\cite{yamazaki2022spiking, snnyolo, s23063037}.
さらに, SNNは入力ノイズに対するロバスト性が高いとされている.
画像認識における入力画像へのガウシアンノイズに対するロバスト性\cite{zhao2022spiking}や, SNNを用いて学習した強化学習エージェントのノイズによるパフォーマンス低下を抑制する結果が報告されている\cite{patel2019improved}.
このような特性から, SNNは次世代のニューラルネットワークとしての関心を集めている\cite{maass1997networks}.
