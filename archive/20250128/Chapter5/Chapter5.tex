\cleardoublepage % 奇数ページから始める
\chapter{結論}

本研究では, 入力速度変化に対する汎化性能の高いSpiking Neural Networkの構築を目的として, 入力速度変化に応じたSNNの時定数および膜抵抗を動的に変化させる手法を提案した.
提案手法の実験的妥当性の検証とジェスチャー分類・マニピュレータ軌道予測問題に適用を行い, その有効性を評価した.

第1章では, SNN電力効率性とノイズロバスト性の高さについて触れ, 深層学習におけるSNNの優位性について述べた.
さらに, SNNが過去情報の記憶に当たる内部状態を持つことを取り上げ, その時系列情報処理における適正性について述べた.
しかしながら, SNNを用いた時系列情報処理を行うにあたって, 内部状態の時間変化を決定する時定数がモデル設計者に依存する問題点があることを指摘した.
この問題に取り組んだ先行研究として, 時定数をニューラルネットワークの重みやバイアスと同様に学習可能とするParametric SNNや, 学習可能な時定数を1つのニューロンに複数割り当てるDH-SNNを紹介した.
一方で, これらの先行研究は, 異なる速度の時系列情報を扱うにはそれぞれの速度を学習する必要があり, その学習効率と汎化性能の低さを課題として指摘した.
そこで, この課題を解決するために, 入力速度変化に対する汎化性能の高いSpiking Neural Networkの構築を目的とした研究を行った.

第2章では, SNNの基本的な概念とその学習方法について説明を行った.
また, 提案手法の考え方とその具体的な定式化について述べた.
最後に, 行った3つの実験方法について述べた.

第3章では, 実験の結果を示した.
1つ目の実験では, 提案手法の妥当性の検証を行った.
結果として, 提案手法を用いることで, 入力のタイムスケール変化に応じて, SNN内部状態を近似的にタイムスケーリングすることが可能であることを示した.
2つ目の実験では, ジェスチャー分類における提案手法の有効性を評価した.
結果として, 提案手法を用いることで, 入力の速度変化に対する分類精度の低下を抑制できることを示した.
3つ目の実験では, マニピュレータ軌道予測における提案手法の有効性を評価した.
結果として, 提案手法を用いることで, 入力の速度変化に対する軌道予測精度の低下を抑制できることを示した.

第4章では, 3章で得られた結果についての議論を行った.
1つ目の実験では, SNNの構造 (Batch Normalization構造) によっては, 内部状態をタイムスケーリングに近似できない場合があった.
そこで, Batch Normalization構造を用いた場合の提案手法を定式化し, タイムスケーリングに近似することが困難であることを示した.
2つ目, 3つ目の実験については, 提案手法の有無によるSNNの内部状態の変化を比較した.
2つ目のジェスチャー認識実験では, 提案手法を用いることで, 入力速度変化による内部状態の活動パターンの変化が抑制され, 分類精度の維持につながることを述べた
また, 3つ目のマニピュレータ軌道予測実験では, 目標軌道速度変化に応じて, SNNの内部状態の周期がスケールされることを示した.
結果として, 提案手法を用いることで, 入力速度変化による軌道予測精度の低下を抑制できることを述べた.

今後の展望として, LIFモデルよりも表現力の高いニューロンモデルに本手法を適用することがあげられる.
LIFモデルは, 膜電位変化を1階の微分方程式で記述したものであり, 最もシンプルなニューロンモデルである.
一方で, ニューロンモデルには, 複数階の微分方程式で記述されるものや, 再帰構造を持つものが存在する.
これらは, LIFモデルモデルと比較して, より複雑な時間表現を記述することが可能である.
そのため, 本手法をこれらのニューロンモデルに適用することで, より複雑な時間表現を獲得しつつ, 本研究で示した効果を持つSNNを構築可能になることが期待される.
