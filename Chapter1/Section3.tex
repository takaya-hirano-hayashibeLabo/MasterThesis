\section{東京進出}
2011年2月2日、3年間の留学を終えて正式に帰国し、芸人に復帰すると共に東京進出の意向を明らかにした[13]。同年12月31日、日本テレビ『ダウンタウンのガキの使いやあらへんで!』の年越企画「笑ってはいけないシリーズ」の『絶対に笑ってはいけない空港24時』に出演。また、2013年には同シリーズの『絶対に笑ってはいけない地球防衛軍24時』内で松方弘樹が演じる防衛軍長官によって松本人志(ダウンタウン)にコードネームとして「まつもときんに君」が使用された。

2012年7月4日、映画『アベンジャーズ』のイベントに出演し、縁の深い八木真澄(サバンナ)の結婚式についての話題から自身が25歳のベルギー人女性と交際中であることを明かした[14]。

2014年5月、5人座長体制の新喜劇で座長を務める1人である小籔千豊の東京新喜劇公演に招集され、日替わりゲストとして出演した[15]。同年には同じく小籔の主催する「KOYABU SONIC 2014 FINAL」にも出演した[16]。

2019年9月、特撮テレビドラマ『仮面ライダーゼロワン』第1話「オレが社長で仮面ライダー」でヒューマギア(人型ロボット)・腹筋崩壊太郎役を演じると共にその怪人態・ベローサマギアの声も演じ、その役名と熱演ぶりから放送当日のTwitterでトレンドに入るほどの話題となった[17]。また、人々を笑わせることに喜ぶ同キャラクターが怪人化させられて悲壮な最期を遂げるという展開にちなむ単語「腹筋崩壊太郎ロス」が合わせてトレンド入りを果たしたことに、なかやまも感謝している[18][19]。後に、自身のYouTubeチャンネル『ザ・きんにくTV 【The Muscle TV】』のサブチャンネルで明かしたところによれば、オファーを受けた当初は驚いて令和初の怪人ということからも演技への不安にかられたが、役名を見て良い意味で肩(僧帽筋)の力が抜け、撮影の際にはバラエティ番組とは違った現場の雰囲気を楽しんだという[20][21]。また、後年のインタビューによれば、『ゼロワン』第1話の放送当日はオールジャパンフィジークの予選に出場していたため、予選中には知人からトレンド1位を知らせるLINEが多々来ていたという[22][2][注 7]。また、腹筋崩壊太郎の決め台詞「腹筋パワー!」や着用していたサスペンダーは当初には無く、前者はリハーサルの際に監督からの依頼に思いつき、後者は衣装合わせを経てそれぞれ決まったという[22][2]。アフレコではいつもネタで言っている「パワー」や「ヤー」が活かされたという[2]。

なお、「腹筋崩壊太郎」はその後もたびたびトレンド入りしており[23][24]、『ゼロワン』第21話に登場した「腹筋崩壊太郎の腹筋」が商品化されたほか[2]、2020年3月28日には短編スピンオフ作品もYouTubeにて1年間限定配信で公開された[25]ほか、次作『仮面ライダーセイバー』(テレビ朝日)でも筋肉アイドルの才木玲佳が出演する際に同作の公式Twitterが「腹筋崩壊太郎枠」という語句を用いたことが報じられている[26]。また、腹筋崩壊太郎は2020年12月18日に公開された映画『劇場版 仮面ライダーゼロワン REAL×TIME』にも登場した[27]ほか、それに先駆けて同年12月7日にはなかやまが腹筋崩壊太郎として男性向け週刊誌『週刊プレイボーイ』(集英社)2020年51号の巻末グラビアを歴代ヒロインたちと共に飾ったことが、珍事として報じられている[28]。その後、2021年12月23日には(なかやまにとっては15年ぶりの出演でもある)新喜劇にて「復活」したことが、『ザ・きんにくTV 【The Muscle TV】』を経て報じられている[29]。

2022年1月25日、2021年12月31日付をもって、なかやま本人の希望で吉本興業とのマネジメント契約を終了していたことを発表した[30][31]。その理由として、海外進出を志していること、お笑いよりもフィットネスの仕事の比重が高くなったことを自ら挙げている[32]。