\section{考察}
\subsection{未学習速度の軌道予測における内部状態のタイムスケーリングによる影響}
節\ref{sec:result3}では, 提案手法を用いたSNNのロボットマニピュレータ軌道予測問題におけるモデル精度評価を行った.
結果として, 提案手法を用いたSNNは未学習の速度に対して, 通常のSNNよりも目標軌道に近い軌道を予測することが可能であった.
この結果は, 提案手法によって入力の速度変化に応じてSNNの内部状態をタイムスケーリングに近似することで, 予測軌道も単純なタイムスケーリングとなったからだと考えられる.
\figref{fig:model:learning:exp3}に示す通り, モデルによる手先位置の変化量の推定は, SNNの内部状態を入力とするNNによって行われる.
ここで, NNは空間方向の非線形変換のみを行う.
そのため, 入力速度変化に伴う, 予測軌道の誤差はSNNの内部状態ダイナミクスの変化に依存すると考えられる.
通常のSNNでは, \figref{fig:discussion3:snn:a3}, \figref{fig:discussion3:snn:a05}より, 目標タイムスケールの変化に伴う, SNN内部状態の周期の変化が小さい結果であった.
これは, 入力速度が変化したのにも関わらず, NNが学習時と変わらない周期の入力を受け取ることを意味する.
入力の速度変化に対して, NNの推定する軌道の周期は変化しないこととなり, 手先の軌道が目標軌道から大きく外れる結果となったと考えられる.
一方で, 提案手法のSNNでは, \figref{fig:discussion3:dyna:a3}, \figref{fig:discussion3:dyna:a05}より, 目標タイムスケールの$a$倍の変化に伴って, SNNの内部状態の周期も$a$倍に変化する結果となった.
これは, 入力の速度変化に伴って, NNが入力の速度変化に応じた周期の入力を受け取ることを意味する.
結果として, 速度変化に伴う入出力の時間方向の誤差が小さくなり, 空間的な予測軌道の誤差が小さく抑えられたと考えられる.
従って, 提案手法を用いることで, 基準速度の軌道のみの学習によって, 未学習の速度変化に対しても, 目標軌道に近い軌道を予測することが可能であったと考えられる.

