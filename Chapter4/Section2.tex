\section{結果} \label{sec:result2}
提案手法をジェスチャー分類問題に適用した際の, 入力速度変化に対するモデル精度評価を行った結果を示す.
ジェスチャー動画全体の速度倍率を変化させた場合と, ジェスチャーの途中で速度倍率を変化させた場合のモデル精度評価を行った.

\subsection{モデル学習結果}
通常のSNN, Parametric-SNN, 提案手法を用いたSNNのモデル学習曲線を\figref{fig:result2:losscurve}, \figref{fig:result2:acccurve}に示す.
それぞれの曲線におけるtrain, testはそれぞれ学習データ, 検証データに対する損失誤差と精度を表す.
全てのモデルにおいて損失曲線, 精度曲線ともに収束しており, モデルの学習は正常に行われた.
また, 入力速度変化に対するモデル精度の検証では, 最もテストデータに対する精度が高いモデルを用いた.
検証に用いたモデルについては\tabref{tab:result2:model:parameter}に示す.
\begin{figure}[htb]
    \centering
    \includesvg[width=1.0\textwidth, inkscapelatex=false]{Static/exp2_losscurve}
    \caption{ジェスチャー分類問題の学習曲線 : 損失曲線}
    \label{fig:result2:losscurve}
\end{figure}
\begin{figure}[htb]
    \centering
    \includesvg[width=1.0\textwidth, inkscapelatex=false]{Static/exp2_acccurve}
    \caption{ジェスチャー分類問題の学習曲線 : 精度曲線}
    \label{fig:result2:acccurve}
\end{figure}

\begin{table}[htb]
    \centering
    \caption{ジェスチャー分類問題の検証に用いたモデル}
    \label{tab:result2:model:parameter}
    \begin{tabular}{cccc}
        \hline
        \textbf{Model}& \textbf{Epoch} & \textbf{Accuracy Mean} & \textbf{Accuracy Std}\\
        \hline
        SNN &  211 & 0.878 & 0.073\\
        Parametric-SNN & 220 & 0.909 & 0.086\\
        Proposed & 222 & 0.920 & 0.086\\
        \hline
    \end{tabular}
\end{table}
\clearpage

\subsection{全体の入力速度変化に対するモデル精度評価}
入力ジェスチャーの各タイムスケール$a$に対するモデル精度の変化を\figref{fig:result2:eval1}に示す.
横軸$a$は入力スパイクのタイムスケール倍率, 縦軸は各タイムスケールにおけるモデル精度を表す.
また, 青色が通常のSNN, 青緑色がParametric-SNN, 黄緑色が提案手法を用いたSNNを表す.
通常のSNN, Parametric-SNNは, タイムスケール$a$が$1.0$から離れると, そのモデル精度が低下していることがわかる.
提案手法は, タイムスケール$a$が$1.0$よりも大きい場合 (入力速度が遅くなる場合) において, そのモデル精度の低下が抑制される結果となった.
また, タイムスケール$a$が$1.0$よりも小さい場合においては, 通常のSNN, Parametric-SNNと比較すると, モデル精度の低下が抑制される結果となった.
\begin{figure}[htb]
    \centering
    \includesvg[width=1.0\textwidth, inkscapelatex=false]{Static/exp2_eval1}
    \caption{ジェスチャー分類問題の入力速度変化に対するモデル精度評価}
    \label{fig:result2:eval1}
\end{figure}


\subsection{途中の入力速度変化に対するモデル精度評価}
途中で入力速度が変化する場合のモデル精度を\figref{fig:result2:eval2}に示す.
各グラフのアルファベットは速度変化のパターンを表す (\tabref{tab:model:parameter:speed:change}).
また, 青色が通常のSNN, 青緑色がParametric-SNN, 黄緑色が提案手法を用いたSNNを表す.
\figref{fig:result2:eval2}より, 提案手法のみ, 学習時の速度の有無に関わらず, モデル精度が維持される結果となった.
これらの結果より, 提案手法は動画分類問題において, 未学習の速度に対するモデル精度の低下を抑制されることが示された.
\begin{figure}[htb]
    \centering
    \includesvg[width=1.0\textwidth, inkscapelatex=false]{Static/exp2_eval2}
    \caption[ジェスチャー分類問題の入力速度変化に対するモデル精度評価]{
        ジェスチャー分類問題の入力速度変化に対するモデル精度評価.
        A: 学習速度 → 低速倍率.
        B: 学習速度 → 高速倍率.
        C: 低速倍率 → 高速倍率.
        D: 高速倍率 → 低速倍率.    
    }
    \label{fig:result2:eval2}
\end{figure}


\subsection{入力速度変化に対するSNNの内部状態変化}
あるジェスチャーをモデルに入力した場合におけるSNNの最終層の内部状態変化を\figref{fig:discussion2:snn:a5} - \figref{fig:discussion2:dyna:a05}に示す.
それぞれの図は, 学習時の速度と比較して, 入力速度が低速倍率$a=5.0$と高速倍率$a=0.5$の場合のSNNの内部状態変化を示す.
横軸は時間を表し, 縦軸は最終層のニューロンのインデックスを表す.
ここで各ニューロンは分類対象のジェスチャーのクラスに対応している.
また, 図における色は内部状態の値の大きさを表し, 最も活性化している (黄色の面積が多い) ニューロンがそのモデルの予測クラスに対応する.

学習時速度$a=1.0$を入力したときの内部状態図 (\figref{fig:discussion2:snn:a5} - \figref{fig:discussion2:dyna:a05}における上段の図) より, 通常のSNNも提案手法のSNNも正解ラベルのニューロンが最も活性化しており, 正しく推論することが可能であった.
しかしながら, 通常のSNNは入力速度変化に従って, 内部状態の活性パターンが大きく変化した.
\figref{fig:discussion2:snn:a5}より, 低速倍率 ($a=5.0$) の入力において, 正解であるクラス3のニューロンの活性化が小さくなり, クラス8のニューロンの活性化が大きくなっている.
また, 高速倍率 ($a=0.5$) の入力において, 正解であるクラス1のニューロンの活性化が小さくなり, クラス3, クラス4のニューロンの活性化が大きくなっている.
一方で, 提案手法のSNNは入力速度変化に伴う, 内部状態の活性パターンの変化が小さい結果となった.
\figref{fig:discussion2:dyna:a5}, \figref{fig:discussion2:dyna:a05}より, 高速倍率, 低速倍率のどちらの場合であっても, 正解ラベルに対応するニューロンが最も活性化し, 他のニューロンにおいてもその変化が小さいことがわかる.
これらのことから, 提案手法を用いることで, 入力速度変化による内部状態の活性パターンの変化が抑制されることが示された.
\begin{figure}[htbp] %画像形式はこのチャプターのreadmeを参照
    \centering
    \includesvg[width=1.0\textwidth, inkscapelatex=false]{Static/disc2_snn5}
    \caption{
        低速ジェスチャー認識における通常のSNN最終層の内部状態変化
        }
    \label{fig:discussion2:snn:a5}
\end{figure}

\begin{figure}[htbp] %画像形式はこのチャプターのreadmeを参照
    \centering
    \includesvg[width=1.0\textwidth, inkscapelatex=false]{Static/disc2_dyna5}
    \caption{
        低速ジェスチャー認識における提案手法を用いたSNN最終層の内部状態変化
        }
    \label{fig:discussion2:dyna:a5}
\end{figure}

\begin{figure}[htbp] %画像形式はこのチャプターのreadmeを参照
    \centering
    \includesvg[width=1.0\textwidth, inkscapelatex=false]{Static/disc2_snn0.5}
    \caption{
        高速ジェスチャー認識における通常のSNN最終層の内部状態変化
        }
    \label{fig:discussion2:snn:a05}
\end{figure}

\begin{figure}[htbp] %画像形式はこのチャプターのreadmeを参照
    \centering
    \includesvg[width=1.0\textwidth, inkscapelatex=false]{Static/disc2_dyna0.5}
    \caption{
        高速ジェスチャー認識における提案手法を用いたSNN最終層の内部状態変化
        }
    \label{fig:discussion2:dyna:a05}
\end{figure}


