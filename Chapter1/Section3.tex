\section{先行研究での課題}
先行研究では, SNNにおける時定数をニューラルネットワークの重みとバイアスと同様に学習可能とすることで, SNNの時間表現力を向上させた.
しかしながら, 入力速度に対する汎化性の面で課題があると考えられる.
時定数の学習による時間表現能力の向上には, 学習時に多様な速度帯のデータが必要である.
このような速度のみが異なるデータは, 時間的特性以外は類似の情報を持つことが多く, その学習効率が低下する.
例えば, ジェスチャー認識のデータセットでは, 速度の高低に関わらず同一のジェスチャーである.
そのため, 入力速度変化に対して頑健な推論を行うために, 基準速度データの学習のみで汎化のSNNを構築することが望ましいと考えられる.
