\section{本論文の構成}
本論文の構成は以下に示す通りである.

\subsubsection{第1章 序論}
本研究の背景および先行研究について述べた.
また, 先行研究での課題とその課題に対する本論文の目的を述べた.

\subsubsection{第2章 手法}
本研究で用いるSpiking Neural Networkの生物学的背景とそのモデル化について述べる.
また, 本研究の提案手法とその導出過程を記述する.
さらに, 提案手法を用いた実験内容について述べる.

\subsubsection{第3章 実験結果}
一般的なネットワーク構造を持つSNNに提案手法を適用したときの妥当性を実験的に検証する.
また, 提案手法をジェスチャー分類問題, マニピュレータ軌道予測問題に適用したときの有効性を評価する.

\subsubsection{第4章 考察}
提案手法の有効性が得られなかったネットワーク構造について定式化を行い, 提案手法の効果を得ることが困難であったことを示す.
また, ジェスチャー分類問題, マニピュレータ軌道予測問題において, 提案手法で良好な結果が得られた原因を, SNNの内部状態の観点から議論する.

\subsubsection{第5章 結言}
本論文での結論, ならびに今後の展望について述べる.
