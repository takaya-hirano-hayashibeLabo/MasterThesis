\section{本論文の構成}
本論文の構成は以下に示す通りである.

\subsubsection{第1章 序論}
本研究の背景および先行研究について述べた.
また, 先行研究での課題とその課題に対する本論文の目的を述べた.

\subsubsection{第2章 手法}
本研究で用いるSpiking Neural Networkの生物学的背景とそのモデル化について述べる.
また, 本研究の提案手法とその導出過程を記述する.

\subsubsection{第3章 入力速度変化に対する内部状態変化の検証}
一般的なネットワーク構造を持つSNNに提案手法を適用したときの妥当性を実験的に検証する.

\subsubsection{第4章 ジェスチャー分類問題における提案手法の速度変化に対する頑健性評価}
ジェスチャー分類問題に提案手法を適用し, 速度変化に対する頑健性を検証する.

\subsubsection{第5章 軌道予測問題における提案手法の速度変化に対する頑健性評価}
軌道予測問題に提案手法を適用し, 速度変化に対する頑健性を検証する.

\subsubsection{第6章 結言}
本論文での結論, ならびに今後の展望について述べる.
