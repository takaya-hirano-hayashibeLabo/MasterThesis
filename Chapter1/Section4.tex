\section{研究目的}
本研究の目的は, 未学習の入力速度に対して汎化なSpiking Neural Networkを構築することである.
この目的のために, 時定数を含むSNNのパラメータを推論時にも動的に変化させる機構を提案する.
先行研究と提案手法の時定数に対するアプローチの違いを\tabref{tab:method:comparison}に示す.
先行研究では, 時定数は学習可能であったが, 推論時にはその学習済みの時定数を固定して使用する.
本研究では, 時定数が学習可能であることに加えて, 推論時には時定数が動的に変化する.
これによって, 未学習の速度帯のデータであっても, SNNの振る舞いが学習時の速度帯が入力された場合と同様になることが期待される.
結果として, 入力速度帯に関係なく汎化なSNNを構築することが可能となる.
本研究では, 提案手法をSNNに組み込み, その原理の実験的検証と実問題への適用を行う.
\begin{table}[htb]
    \centering
    \caption[先行研究と提案手法の時定数に対するアプローチ比較]{
        先行研究と提案手法の時定数に対するアプローチ比較. 提案手法のみ推論時の時定数が動的に変化する.
    }
    \label{tab:method:comparison}
    %{\small %12ptだとはみ出るので小さく
    \begin{tabular}{cccc}
        \hline
         & \textbf{従来のSNN} & \textbf{先行研究}\cite{dhsnn,paramsnn} & \textbf{提案手法}\\
        \hline
        事前設定 & 固定 & なし & なし\\
        学習時 & 学習不可 & 学習可能 & 学習可能\\
        推論時 & 固定 & 固定 & \textbf{動的}\\
        \hline
    \end{tabular}
    %}
\end{table}
