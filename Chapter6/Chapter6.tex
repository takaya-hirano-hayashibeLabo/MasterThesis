\cleardoublepage % 奇数ページから始める
\chapter{結論}

本研究では, 入力速度変化に対する汎化性能の高いSpiking Neural Networkの構築を目的として, 入力速度変化に応じたSNNの時定数および膜抵抗を動的に変化させる手法を提案した.
提案手法の実験的妥当性の検証とジェスチャー分類・マニピュレータ軌道予測問題に適用を行い, その有効性を評価した.

第1章では, SNN電力効率性とノイズロバスト性の高さについて触れ, 深層学習におけるSNNの優位性について述べた.
さらに, SNNが過去情報の記憶に当たる内部状態を持つことを取り上げ, その時系列情報処理における適正性について述べた.
しかしながら, SNNを用いた時系列情報処理を行うにあたって, 内部状態の時間変化を決定する時定数がモデル設計者に依存する問題点があることを指摘した.
この問題に取り組んだ先行研究として, 時定数をニューラルネットワークの重みやバイアスと同様に学習可能とするParametric SNNや, 学習可能な時定数を1つのニューロンに複数割り当てるDH-SNNを紹介した.
一方で, これらの先行研究は, 異なる速度の時系列情報を扱うにはそれぞれの速度を学習する必要があり, その学習効率と汎化性能の低さを課題として指摘した.
そこで, この課題を解決するために, 入力速度変化に対する汎化性能の高いSpiking Neural Networkの構築を目的とした研究を行った.

第2章では, SNNの基本的な概念とその学習方法について説明を行った.
また, 提案手法の考え方とその具体的な定式化について述べた.

第3章では, 提案手法の入力速度変化に対する内部状態変化の検証を行った.
結果として, Linear, CNN, Dropout, ResNet構造において, 提案手法を用いることで, 入力速度変化に対する内部状態ダイナミクスの変化を抑制できることを示した.
しかし, Batch Normalization構造においては, 提案手法による内部状態変化を抑制することが困難であった.
この原因は, Batch Normalization構造における標準化操作が, 提案手法のバイアスが0でなくてはならない条件と反しており, 提案手法の有効性が発揮できないからであると考察した.

第4章では, ジェスチャー分類問題に対して提案手法を適用し, 入力速度変化に対するモデル精度を評価した.
結果として, 提案手法を用いることで, 入力速度変化に対する分類精度の低下を抑制できることを示した.
提案手法を用いることで, 入力速度変化時の内部状態の活動パターンの変化が抑制され, 分類精度の維持につながると考えられる.

第5章では, マニピュレータ軌道予測問題に対して提案手法を適用し, 目標速度変化に対する軌道予測精度を評価した.
結果として, 提案手法を用いることで, 入力速度変化に対する軌道予測精度の低下を抑制できることを示した.
提案手法を用いることで, 目標軌道速度変化に応じて, SNNの内部状態の周期がスケールされることを示した.
これによって, モデルの予測軌道も目標速度に合わせて加減速が可能であったと考えられる.

今後の展望として, LIFモデルよりも表現力の高いニューロンモデルに本手法を適用することがあげられる.
LIFモデルは, 膜電位変化を1階の微分方程式で記述したものであり, 最もシンプルなニューロンモデルである.
一方で, ニューロンモデルには, 複数階の微分方程式で記述されるものや, 再帰構造を持つものが存在する.
これらは, LIFモデルモデルと比較して, より複雑な時間表現を記述することが可能である.
そのため, 本手法をこれらのニューロンモデルに適用することで, より複雑な時間表現を獲得しつつ, 本研究で示した効果を持つSNNを構築可能になることが期待される.
