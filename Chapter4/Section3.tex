\section{考察}
\subsection{ジェスチャー分類問題における動的な時定数の影響}
第\ref{sec:result2}節では, 提案手法を用いてジェスチャー分類問題を扱い, 未学習速度に対する分類精度を検証した.
その結果, 提案手法を用いたSNNは, 通常のSNNやParametric-SNNと比較して, 入力速度変化に対してモデル性能が維持されることが示された.
また, 提案手法を用いることで, 入力速度変化に伴うSNNの内部状態の活性パターンを抑制できることが示された.
これは提案手法によって, 動的に時定数を変化させることで, 入力速度に応じてSNNの内部状態の変化量を調整することが可能であったからだと考えられる.
入力ジェスチャー動画の速度が変化すると, 1ステップあたりの情報量は変化する.
通常のSNNやParametric-SNNの場合, 内部状態の変化量は学習に用いたデータの速度に対して, モデルのパラメータは最適化され固定されている.
そのため, 入力の1ステップあたりの情報量が変化した場合であっても, 内部状態の変化量は学習時の速度においての最適な変化となる.
このように, 推論時の時定数が静的なSNNは, 入力の情報量変化に対して, その膜電位変化が対応することが困難である.
結果として, 未学習速度のジェスチャー動画が入力された場合, その内部状態の活性パターンは学習時のものから大きく変化し, 誤った推論を行ったと考えられる.
一方で, 提案手法を用いたSNNは, 入力速度変化に応じて時定数を動的に変化させることが可能である.
ここで, 提案手法における時定数は入力の速度が遅くなるとその値は大きくなるように変化する.
これは, 内部状態の時間経過に伴う忘却率が小さくなることを意味する.
よって, 提案手法は入力の低速倍率化に対して, その内部状態の変化量は減少する.
これは, 入力の低速倍率化に伴う1ステップあたりの情報流入量の減少に対して, 内部状態をより長期間維持しようとする働きであると考えられる.
速度の高速倍率化では, その逆の特性を示す.
このように, 提案手法を用いることで, 入力の情報量変化に対して, その膜電位の変化量を調整することが可能である.
結果として, 未学習の入力速度に対しても, SNNの内部状態が学習時と同様のパターンで振る舞い, モデルの分類精度の維持に寄与したと考えられる.
\clearpage