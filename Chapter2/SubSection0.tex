\subsection{生物学的背景}
Spiking Neural Network (SNN)は, 生物の脳神経回路を模倣した数理モデルである.
脳神経回路は, ニューロンと呼ばれる細胞体, 樹状突起, 軸索, シナプスからなる要素の接続によって構成される.
無数のニューロンによって, 入力された電気パルスが出力信号に変換されることで, 生物のような複雑なダイナミクスを表現することが可能となる.

典型的なニューロンの構成要素について, \figref{fig:neuron}に示す.
まず, 細胞体は数マイクロメートルから数十マイクロメートルほどの大きさ, 樹状突起の長さは数十マイクロメートルから1ミリメートル程度である.
軸索の長さは長いもので1メートル近くに達する.
ニューロンにおける情報は, 樹状突起から細胞体, 軸索を通してシナプスに向かい, 他のニューロンへと伝達される.
まず樹状突起は, 前に接続されたニューロンのシナプスから信号を受け取る.
この信号によって, 樹状突起や細胞体の中の内部状態である膜電位が変化する.
このとき, 膜電位を正の方向に変化させる信号を興奮性シナプス, 負の方向に変化させる信号を抑制性シナプスと呼ぶ.
興奮性シナプスと抑制性シナプスによる刺激によって, 膜電位があるしきい値を超えると, 膜電位が急激に上昇する.
この電圧は活動電位と呼ばれ, パルス状の信号として軸索上を伝達する.
伝達速度は0.5 m/sから100 m/s程度であり, 軸索の種類や太さによって大きく差がある.
軸索の終端に電気信号が到達すると, シナプスから化学伝達物質が分泌され, 接続された樹状突起や細胞体の膜電位を変化させる.
このようなニューロンをネットワーク状に接続し, パルス状の信号を伝達することで, 脳神経回路は複雑な情報処理を行っている.

\begin{figure}[htbp]
    \centering
    \includesvg[width=1.0\textwidth, inkscapelatex=false]{dummy/dummy_img}
    \caption{ニューロンの構成要素}
    \label{fig:neuron}
\end{figure}
