\section{筋肉留学}
\subsection{一度目の留学}
2006年10月9日(体育の日)に芸能生活を一時休業し、一年間の予定でボディビルダーにとっての聖地であるロサンゼルスへ向かい、自らのトレードマークである肉体を強化するための"筋肉留学"を行った。

2007年2月9日(2月9日の金曜日、筋肉にかけている)カリフォルニア州サンタモニカで全編英語のライブを行った。当初一回公演の予定だったものの現地の観客が多数詰め掛けたため急遽二回公演に変更し、およそ100人ほどの聴衆を前にライブを披露した。現地では「nobody knows your comedic style(前人未踏の芸風)」と評され、「賞賛を得た」と報道されている[6]。キャプテン☆ボンバーも披露し、日本とは逆に日本の文化をネタにしていた。2007年8月31日(現地時間)、初めて自らが主演・監督・脚本を担当した35分の自主製作映画「キャプテン☆ヒーロー」を公開。

2007年9月7日、予定期間を終えて筋肉留学から帰国。帰国した模様が同日に放送の毎日放送『ちちんぷいぷい』で取り上げられた。その翌日には毎日放送『せやねん!』に生出演。帰国後初のテレビ番組出演で、3日後の9月10日には読売テレビ『なるトモ!』にも生出演した。9月29日には『せやねん!』に帰国後2回目の生出演、レギュラー番組に復帰した。2007年のM-1グランプリのために同じく筋肉質の芸人八木真澄(サバンナ)と漫才コンビ「ザ☆健康ボーイズ」を結成し、準決勝進出の好成績を挙げた。2008年も「ザ☆健康ボーイズ」として、M-1に再挑戦[7]しており、現在もそれぞれの活動と並行して不定期でのコンビ活動を継続している[注 5]。同年には日本テレビ『ダウンタウンのガキの使いやあらへんで!!』で筋肉料理研究家マグマ中山として出演し、以降も定期的にこのネタを行っている。

% svgのままぶち込む
\begin{figure}[htbp]
  \centering
  \includesvg[width=0.5\textwidth, inkscapelatex=false]{Chapter1/figures/kinniku}
  \caption{筋肉SVG}
  \label{fig:my_label}
\end{figure}

% Learning Based
\subsection{二度目の留学}
帰国後の芸能活動を順調に再開しつつあった2008年1月23日、「1年間程度では大した成果が出ない」との考えから今度は永住覚悟、あるいは無期限での筋肉留学を開始した。アメリカでは一度目のように肉体強化だけではなく、語学勉強や空手ジムに通うなどの本格的な留学計画を立てている[7]。渡米後、カルフォルニア州の2年制の公立大学であり、アーノルド・シュワルツェネッガーの母校でもあるサンタモニカ・カレッジに留学した。

再留学後も定期的に帰国しており、留学40日後の2008年3月3日にはビザの申請のため急遽帰国し、改めてアメリカに向かっていた[8][9][注 6]。3月27日発売の「モンスターハンターポータブル 2nd G」のCM「芸人編」に起用された他、M-1グランプリへの再挑戦(詳細は前述)のため、10月29日にも一時帰国している。日本滞在中は「LIVE STAND 08 OSAKA」への出演に加え、留学前まで準レギュラーだった『せやねん!』をはじめ、フジテレビ『爆笑レッドカーペット』、TBS『ウンナン極限ネタバトル! ザ・イロモネア 笑わせたら100万円』などにも出演した。同年12月末に再渡米。留学の目的の一つとして「アメリカの映画やコメディショーに出演したい」という夢があり実際オーディションにもいくつか合格していたものの、留学ビザしか取得していなかったため出演は叶わなかったという[10]。

2010年7月、先輩のなだぎ武がブログ上で「留学前より線が細くなった」と写真付きで報告しており[11]、むしろ筋力的にも衰えた状態となっていた。2011年1月、サンタモニカ・カレッジ運動生理学部を卒業した[12]。成績はGPA3.38と好成績を収めており、授業も日常会話も全て英語の環境で真面目に授業を受けていたことが明らかになり、憧れのシュワルツェネッガーの名も正確に発音できる(本人談)レベルの語学力を獲得した。卒業を記念してレイザーラモンHG(レイザーラモン)をゲストに「Muscle Comedy 3 〜筋肉留学卒業公演〜」をキャンパス内で開催した。