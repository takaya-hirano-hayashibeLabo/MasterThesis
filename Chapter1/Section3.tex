\section{先行研究での課題}
先行研究では, SNNにおける時定数をニューラルネットワークの重みとバイアスと同様に学習可能とすることで, SNNの時間表現力を向上させた.
しかしながら, 入力速度に対する汎化性の面で課題があると考えられる.
時定数の学習による時間表現能力の向上には, 学習時に多様な速度帯のデータを与える必要がある.
これは, 速度倍率の変化によって, データのタイムステップ間の時間関係が変化するため, それぞれの速度域に対応可能な時定数を学習する必要があるためである.
しかしながら, このような速度のみが異なるデータは時間的特性以外は類似の情報を持つことが多く, その学習効率が低下する.
例えば, ジェスチャー認識のデータセットでは, 速度の高低に関わらず同一のジェスチャーである.
そのため, 入力速度変化に対して頑健な推論を行うために, 基準速度データの学習のみで汎化性の高いSNNを構築することが望ましいと考えられる.
